\documentclass[12pt]{article}

\usepackage[utf8]{inputenc}
\usepackage{latexsym,amsfonts,amssymb,amsthm,amsmath}
\usepackage{graphicx}

\graphicspath{ {./images/} }

\setlength{\parindent}{0in}
\setlength{\oddsidemargin}{0in}
\setlength{\textwidth}{6.5in}
\setlength{\textheight}{8.8in}
\setlength{\topmargin}{0in}
\setlength{\headheight}{18pt}



\title{Home assignment for Computer Network EDA387}
\author{Group 24}

\begin{document}

\maketitle

\vspace{0.5in}

\paragraph{Problem}

Consider a set \(P=\{p_0,\ldots,p_{n-1}\}\) of \(n\) processors, such that each processor \(p_i\)
is associated with two registers \(r_i\) and \(s_i\) (both of constant size independent of \(n\)).
Processor \(p_i\) cannot read the value in \(s_i\), however, any other processor can. Processor \(p_i\)
has both read and write access rights to \(r_i\) and any other processor has only read-only
access rights to \(r_i\). The value in \(s_i\) is unknown to \(p_i\). The other processors can help \(p_i\)
discover this unknown value. For example, suppose that \(n=2\). Processor \(p_{(i+1)\bmod 2}\)
can write the value of \(s_i\) to \(r_{(i+1)\bmod 2}\) and then \(p_i\) can discover \(s_i\)'s value by reading
\(r_{(i+1)\bmod 2}\). Please solve the problem, which is to let \(p_i\) discover the secret \(s_i\),
for the case in which \(n>2\) is any finite known value.

\subsection*{Question 1}

Assume that all processors have access to a globally unique identifier
within the range $\{0, \ldots, n-1\}$, i.e., the identifier is defined by a one-to-one function
$\mathrm{ID}: P \to \{0, \ldots, n-1\}$. For instance, $\forall p_i \in P: \mathrm{ID}(p_i)=i$, i.e., the identifier of
processor $p_i$ is equal to its index $i$. In practice, MAC addresses are generally assumed to
be globally unique. Is there a solution when processors have globally unique identifiers?
Prove your claims.

\bigskip

\begin{proof}

The answer for this question is \emph{cyclical}. 
Each processor can write the secret of its next processor and so on, 
until the last one which writes the secret from the first processor.

Let the processor $p_i$ be the selected one. 
It has two registers: $r_i$ and $s_i$. 
The process to get the value of $s_i$ can be divided into two stages: 
\textbf{Discovery} and \textbf{Read}.

\begin{itemize}
    \item \textbf{Discovery.}  
    Processor $p_i$ can read the value in the register $s_{(i-1)\bmod n}$ 
    and write it to $r_i$.
    
    \item \textbf{Read.}  
    When the discovery phase is completed, 
    processor $p_i$ can read the value in $r_{(i+1)\bmod n}$.  
    By construction, $r_{(i+1)\bmod n}$ contains $s_i$, 
    so $p_i$ learns its own secret.
\end{itemize}

As stated before, only \(p_i\) has the right to modify the content inside register \(r_i\),
after the discovery step, inside \(r_i\) of any \(p_i\), there will be the value of \(s_{(i-1)\bmod n}\).
Therefore, any processor \(p_i\) can read the content inside the register \(r_{(i=1)\bmod n}\) to obtain the value of \(s_i\).
For each processor, only itself has write-privilege to \(r_i\). Moreover, other processors can only read \(r_i\) and \(s_i\), hence,
the content can only be modified by \(p_i\). This guarantees the content of \(s_i\) that \(p_i\) discovered through neighbor is correct.
With this approach, each processor has to do two operations (discover - reading from others and reading - obtaining its own secret from others) with complexity of O(n) each.

\includegraphics[scale=0.45]{images/assignment1_question1.png}



\end{proof}

\vspace{2in} %Leave space for comments!


\subsection*{Question 2}
Now, suppose processors only have access to \emph{locally unique identifiers}.
Specifically, let $N(p_i) \subseteq P \setminus \{p_i\}$ denote the neighborhood of processor 
$p_i \in P$, which is the set of all processors directly connected to $p_i$.
For each $p_i \in P$, the local identifier is defined by a one-to-one function 
$\mathrm{ID}_i: P \to \{0, \ldots, n-1\}$. 
Note that $\mathrm{ID}_i()$ depends on $p_i$, whereas $\mathrm{ID}()$ is identical for all processors. 

Furthermore, for processors $p_i, p_j, p_k \in P$, 
it may be the case that $\mathrm{ID}_i(k) \neq \mathrm{ID}_j(k)$. 
For example, port numbers are unique to the host but not across the Internet. 

Is there a solution when processors have only locally unique identifiers? 
Prove your claims.

\bigskip

\begin{proof}

Considered the selected processor \(p_i\)

Since the communication graph is complete, every processor is connected to every 
other processor (i.e., each processor is a neighbor of all others). Processor \(p_i\) 
also has its neighbor set.

\begin{itemize}
    \item Each processor will write down each other's secrets as in the \textbf{Question 1}. When 
    this process is done, any processor can inspect its own \(r_i\) register to know that every 
    secrets are present, \emph{except one}. That missing information must therefore be its own
    \(s_i\).
    \item More generally, processors can use a distributed algorithm to discover the network topology. 
    With these information, each processor \(p_i\) must find a path to the processor that can write its
    secret \(s_i\) to some register that \(p_i\) itself can read.
    \item In this scenario, only \emph{local identifiers} are available, processors must determine the network
    structure first, this, use this information to coordinate the sharing of secrets
    \item With the neighbor set for each processor and the exchange of local identifiers, processor can learn the complete
    network topology.
\end{itemize}

\includegraphics[scale=0.25]{images/assignment1_question2.jpg}


\end{proof}


\vspace{2in} %Leave more space for comments!

\vspace{3in}

\end{document}
