\documentclass[12pt]{article}

\usepackage[utf8]{inputenc}
\usepackage{latexsym,amsfonts,amssymb,amsthm,amsmath}
\usepackage{graphicx}
\usepackage{listings}
\usepackage{xcolor} 

\graphicspath{ {./images/} }

\setlength{\parindent}{0in}
\setlength{\oddsidemargin}{0in}
\setlength{\textwidth}{6.5in}
\setlength{\textheight}{8.8in}
\setlength{\topmargin}{0in}
\setlength{\headheight}{18pt}

\lstset{
  mathescape,
  backgroundcolor=\color{gray!10},  
  basicstyle=\ttfamily,
  columns=fullflexible,
  breakatwhitespace=false,      
  breaklines=true,                
  captionpos=b,                    
  commentstyle=\color{green}, 
  extendedchars=true,              
  frame=single,                   
  keepspaces=true,             
  keywordstyle=\color{blue},      
  language=c++,                 
  numbers=none,                
  numbersep=5pt,                   
  numberstyle=\tiny\color{blue}, 
  rulecolor=\color{gray},        
  showspaces=false,               
  showstringspaces=false,
  showtabs=false,                 
  stepnumber=5,                  
  stringstyle=\color{red},   
  tabsize=3,                      
  title=\lstname                
}



\title{Peer review on assignment 3}
\author{Group 24}

\begin{document}

\maketitle

\vspace{0.5in}

\paragraph{Review}

\subsection{Question 1}

\textbf{Overview}

Your solution is correct.

\begin{itemize}
    \item \textbf{Our opinion} this is a great solution, it not only achieved the correct solution but also has a 
    very well-structued pseudo code. 
    \item \textbf{Structure} you have done a great work. highlighting all the important points with title. Additionally,
    the way you used "closure", "stabilization time" made it very close to a "scientific" publication.
    \item \textbf{Major point} your pseudo code is quite long and it resembles an actuall programming language, sometime
    can confused the others. 
\end{itemize}

\subsection{Question 2}

\textbf{Overview}

Your solution is very similar to our solution (and also the solution that Elad showed to the class). Every leaves of this tree
set the size to 1, then they send this information to the parent, to their parent parent,... This creates a layer of correction
that will overwrite the garbage values (if any).

\begin{itemize}
    \item \textbf{Your solution} the solution is easy to understand, but the pseudo code is not. We have to read the closure and
    convergence first to understand what you intended to say.
    \item \textbf{Structure} well-structured, as we stated ealier in the question 1.
    \item \textbf{Improvement} if you can give some drawings or illustrations to this assignment, we think it will be better because
    this problem and the algorithm to solve it are hard to imagine.
\end{itemize}

\end{document}
% vim: set ts=4 sw=4 et:                