\documentclass[12pt]{article}

\usepackage[utf8]{inputenc}
\usepackage{latexsym,amsfonts,amssymb,amsthm,amsmath}
\usepackage{hyperref}
\usepackage{graphicx}
\usepackage{listings}
\usepackage{xcolor} 

\graphicspath{ {./images/} }

\setlength{\parindent}{0in}
\setlength{\oddsidemargin}{0in}
\setlength{\textwidth}{6.5in}
\setlength{\textheight}{8.8in}
\setlength{\topmargin}{0in}
\setlength{\headheight}{18pt}

\lstset{
  mathescape,
  backgroundcolor=\color{gray!10},  
  basicstyle=\ttfamily,
  columns=fullflexible,
  breakatwhitespace=false,      
  breaklines=true,                
  captionpos=b,                    
  commentstyle=\color{green}, 
  extendedchars=true,              
  frame=single,                   
  keepspaces=true,             
  keywordstyle=\color{blue},      
  language=c++,                 
  numbers=none,                
  numbersep=5pt,                   
  numberstyle=\tiny\color{blue}, 
  rulecolor=\color{gray},        
  showspaces=false,               
  showstringspaces=false,
  showtabs=false,                 
  stepnumber=5,                  
  stringstyle=\color{red},   
  tabsize=3,                      
  title=\lstname                
}



\title{EthicsAssignment MPCSC 2025}
\author{TBA}

\begin{document}

\maketitle

\vspace{0.5in}

\paragraph{Question}

On June 9, 2011, Google released a "doodle" honoring \textbf{Les Paul}, which users quickly found addictive to play with. 
A third party, \textit{RescueTime}, estimated that approximately \textbf{5.3 million hours} were spent playing this interactive game --- 
equivalent to nearly eight human lifetimes. \\
\url{http://elgoog.im/guitar/}

\vspace{0.5em}

\begin{enumerate}
    \item Did the doodle make a positive contribution to the world?
    \item Do engineers at Google have an obligation to consider this question before releasing the feature?
    \item What principle(s) should they use to determine the answer?
\end{enumerate}

\paragraph{Notable contribution}

The Doodle is a major successfull from engineers of Google. According to the statistics, it only needs 48 hours
to achieve a 5.1 years of worth of music by the records from U.S users (citation needed). A software product can be considered as meaningful if it spread positive value, in this case, at least the history
of music (which is as old as human history) is lengthened by amateur "composers" just by playing with a software.
Further more, this doodle is originally made in honoring of Les Paul, a famous musician and an inventor, who shaped the
modern electronic guitar. This work achieved both the "entertainment" and the "education" purpose. The year 2011 is the 
era of mobile game, hence there were many arcade game: Angry Bird, Fruit Ninja,\dots but they just simple comsumed people's
time and phone's battery. Meanwhile, this Easter Egg spread the information as well as positive engergy by encouraging 
users to create and share their creation. Unlike gaming products from corps or companies, which always require players to
spend money and time in an unobligated way (to keep your character strong, to obtain rare items,\dots), this mini game costed
nothing. Even the estimation from ResureTime: nearly eight human lifetimes was spent, it is still a helpful one. At least,
people sent music into the universe, not "games".

\paragraph{Obligation}

The answer is yes, engineers must consider the question about contribution of a product before releasing one. Because
product is born to server human needs and it also must not harm any people in any ways. That is the code of ethics for
engineers. There is one truth about humanbeing curiousity: we always want to figure out every secrets of the nature and
even to control the nature. That motivation helps civilization to achieve masterpieces time to time: from the first plane,
the first landing on moon and now, we are reaching closer to an era that humans and intelligence machines can work together
for a better future. But during that process, many experiments was misconducted, leading to unnecessary harm and suffering.
One of the famous example is "The Stanford Prison Exepriment", a psychological experiment to examine the effects of situational
variables on participants's reactions and behaviors. This experiment indeed contributed to the study of psychology, 
especially on how people react with surroundings (environment, authority,\dots). But during this simulation, the level of
violence was uncontrollable, hence it was canceled on the 6th day. This case is a clear example about ethics in reseach, the
condutor Zimbardo did not allow any participants to leave the experiment although they were told that they could. That's 
psychology, but software is not excluded since we are living in a digital society, every day activities are tied to the 
internet and of course, the software. This virtual product can lead to generations that are separated from real world,

\paragraph{Principles}

There are several priciples they should use to determine the answer. The first thing is public. Because engineers shall act
consistently with the public interest, the created product must lie closely in the "interest" zone of 
majority. Next is "product", since any released software should be distributed to user at its finnest from of completion to
meet the highest professional standards as possible. A flaw product might harm users, as the infamous "Therac-25" with a 
programming error (race condition) that resulted in giving patients radiation doses that should not be comsume (hudreds time
greater than acceptable level).
\end{document}
% vim: set ts=4 sw=4 et:s