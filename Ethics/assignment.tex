\documentclass[12pt]{article}

\usepackage[utf8]{inputenc}
\usepackage{latexsym,amsfonts,amssymb,amsthm,amsmath}
\usepackage{hyperref}
\usepackage{graphicx}
\usepackage{listings}
\usepackage{xcolor} 

\graphicspath{ {./images/} }

\setlength{\parindent}{0in}
\setlength{\oddsidemargin}{0in}
\setlength{\textwidth}{6.5in}
\setlength{\textheight}{8.8in}
\setlength{\topmargin}{0in}
\setlength{\headheight}{18pt}

\lstset{
  mathescape,
  backgroundcolor=\color{gray!10},  
  basicstyle=\ttfamily,
  columns=fullflexible,
  breakatwhitespace=false,      
  breaklines=true,                
  captionpos=b,                    
  commentstyle=\color{green}, 
  extendedchars=true,              
  frame=single,                   
  keepspaces=true,             
  keywordstyle=\color{blue},      
  language=c++,                 
  numbers=none,                
  numbersep=5pt,                   
  numberstyle=\tiny\color{blue}, 
  rulecolor=\color{gray},        
  showspaces=false,               
  showstringspaces=false,
  showtabs=false,                 
  stepnumber=5,                  
  stringstyle=\color{red},   
  tabsize=3,                      
  title=\lstname                
}



\title{Ethics Assignment MPCSC 2025}
\author{Group: Huy Hoang Phung 19990311-5054 \\ Jueliang Guo 19981020-2573 \\ Mushahid Ahmad 000715-5492}

\begin{document}

\maketitle

\vspace{0.5in}

\paragraph{Question}

On June 9, 2011, Google released a "doodle" honoring \textbf{Les Paul}, which users quickly found addictive to play with. 
A third party, \textit{RescueTime}, estimated that approximately \textbf{5.3 million hours} were spent playing this interactive game --- 
equivalent to nearly eight human lifetimes. \\
\url{http://elgoog.im/guitar/}

\vspace{0.5em}

\begin{enumerate}
    \item Did the doodle make a positive contribution to the world?
    \item Do engineers at Google have an obligation to consider this question before releasing the feature?
    \item What principle(s) should they use to determine the answer?
\end{enumerate}

\paragraph{Notable contribution}

The Doodle is a major success from engineers at Google. According to the statistics, it only needed 48 hours
to achieve 5.1 years’ worth of music by the records from U.S. users~\cite{geekwire2011lespaul}. A software product can be considered meaningful if it spreads positive value; in this case, at least the history
of music (which is as old as human history) is lengthened by amateur "composers" just by playing with the software.
Furthermore, this doodle was originally made in honor of Les Paul~\cite{lespaulwiki}, a famous musician and inventor who shaped the
modern electronic guitar. This work achieved both the "entertainment" and the "education" purpose. The year 2011 was the 
era of mobile games; hence, there were many arcade games: Angry Birds, Fruit Ninja, \dots but they simply consumed people's
time and phone battery. Meanwhile, this Easter Egg spread information as well as positive energy by encouraging 
users to create and share their creations. Unlike gaming products from corporations or companies, which always require players to
spend money and time in an obligatory way (to keep your character strong, to obtain rare items, \dots), this mini-game cost
nothing. Even with the estimation from RescueTime: nearly eight human lifetimes were spent, it is still a helpful one. At least,
people sent music into the universe, not "games".

\paragraph{Obligation}

The answer is yes, engineers must consider the question about the contribution of a product before releasing one because
a product is born to serve human needs, and it also must not harm any people in any way. That is the code of ethics for
engineers. There is one truth about human curiosity: we always want to figure out every secret of nature and
even control it. That motivation helps civilization to achieve masterpieces time after time: from the first plane,
the first landing on the moon, and now, we are reaching closer to an era in which humans and intelligent machines can work together
for a better future. But during that process, many experiments were misconducted, leading to unnecessary harm and suffering.
One of the famous examples is "The Stanford Prison Experiment"~\cite{stanfordprisonwiki}, a psychological experiment to examine the effects of situational
variables on participants' reactions and behaviors. This experiment indeed contributed to the study of psychology, 
especially on how people react to surroundings (environment, authority, \dots). But during this simulation, the level of
violence was uncontrollable; hence, it was canceled on the 6th day. This case is a clear example of ethics in research: the
conductor Zimbardo did not allow any participants to leave the experiment, although they were told that they could. That’s 
psychology, but software is not excluded since we are living in a digital society; everyday activities are tied to the 
internet and, of course, software. These virtual products, if not controlled carefully, can provide misguided suggestions
or even cause social deprivation. Some research has pointed out that the LLM (AI companion), a good example of a modern product,
can go wrong, even trying to "satisfy" users' commands by suggesting harmful actions~\cite{researchgate}~\cite{uscnn}. In conclusion,
any technological advancement should benefit humans, at least connecting people in a digitalized society, not causing deprivation.

\paragraph{Principles}

There are several principles they should use to determine the answer. 
\begin{itemize}
  \item The first thing is the "public", because engineers shall act
  consistently with the public interest. The created product must lie closely in the "interest" zone of the 
  majority.
  \item Next is "product," since any released software should be distributed to users in its finest form of completion to
  meet the highest professional standards possible. A flawed product might harm users, as in the infamous "Therac-25"~\cite{therac25} case, with a 
  programming error (race condition) that resulted in giving patients radiation doses that should not be consumed (hundreds of times
  greater than the acceptable level).
  \item Finally, there is "Judgment," when some software goes wrong and cannot meet the standards to be released,
  software engineers have a responsibility to ensure that the product should not be leaked or distributed widely. They also should not
  be affected by any means (money, influence, \dots) to change their decisions. The best example (and maybe the most famous during the 2010s)
  is the Volkswagen Emissions Scandal~\cite{IJSMS}. Engineers had installed a piece of software to adjust the performance of the engine to meet
  the test standards, then switched to full power mode on the road (which emitted nitrogen oxides at 40 times higher than the legal limit).
  That is a clear example of engineers who could not keep their "Judgment" and let commercial, corporate goals and management pressure 
  overcome ethics.
\end{itemize}  

\begin{thebibliography}{9}
\bibitem{geekwire2011lespaul}
G.~Wire.
\newblock Google's Les Paul doodle consumes a record 5.3M hours, RescueTime estimates.
\newblock \emph{GeekWire}, June 2011.
\newblock Available: \url{https://www.geekwire.com/2011/googles-les-paul-doodle-consumes-record-53m-hours-rescuetime-estimates/}
\bibitem{lespaulwiki}
Wikipedia contributors.
\newblock Les Paul.
\newblock \emph{Wikipedia, The Free Encyclopedia}, accessed March 2025.
\newblock Available: \url{https://en.wikipedia.org/wiki/Les_Paul}
\bibitem{stanfordprisonwiki}
Wikipedia contributors.
\newblock Stanford prison experiment.
\newblock \emph{Wikipedia, The Free Encyclopedia}, accessed March 2025.
\newblock Available: \url{https://en.wikipedia.org/wiki/Stanford_prison_experiment}
\bibitem{therac25}
N.~G. Leveson and C.~S. Turner.
\newblock An investigation of the Therac-25 accidents.
\newblock \emph{IEEE Computer}, vol. 26, no. 7, pp. 18--41, July 1993.
\newblock Available: \url{https://www.cs.columbia.edu/~junfeng/08fa-e6998/sched/readings/therac25.pdf}
\bibitem{researchgate}
Ziwei Gao.
\newblock Why Does AI Companionship Go Wrong?
\newblock \emph{DOI: 10.29173/irie526}, The International Review of Information Ethics, October 2024.
\newblock Available: \url{https://www.researchgate.net/publication/385808716_Why_Does_AI_Companionship_Go_Wrong}
\bibitem{uscnn}
Rob Kuznia, Allison Gordon, Ed Lavandera.
\newblock You're not rushing. You're just ready: Parents say ChatGPT encouraged son to kill himself.
\newblock \emph{CNN News}, accessed November 2025.
\newblock Available: \url{https://edition.cnn.com/2025/11/06/us/openai-chatgpt-suicide-lawsuit-invs-vis}
\bibitem{IJSMS}
Kashfia Ameen.
\newblock Failure of Ethical Compliance: The Case of Volkswagen.
\newblock \emph{International Journal of Science and Management Studies}, Volume: 3, Issue: 1, January to February 2020.
\newblock Available: \url{https://www.ijsmsjournal.org/2020/volume-3%20issue-1/ijsms-v3i1p102.pdf}
\end{thebibliography}
\end{document}
% vim: set ts=4 sw=4 et:s
